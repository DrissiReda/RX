\documentclass[a4paper,sans,titlepage,10pt]{article}

\usepackage[utf8]{inputenc}
\usepackage{amsmath}
\title{Projet RX}

\author { \textbf{Mohamed Reda DRISSI }
			\and
		   \textbf{Simon ROUVEL}}
\date{\textit{27 janvier 2018}}

\begin{document}

\maketitle

\tableofcontents
\clearpage
\section{Introduction}
Pour simuler tous les cas nous avons cherché les cas possibles :
\begin{itemize}
	\item 1 RRH dans le premier BBU, 5 RRH dans le deuxième BBU
	\item 2 RRH dans le premier BBU, 4 RRH dans le deuxième BBU
	\item 3 RRH dans le premier BBU, 3 RRH dans le deuxième BBU
	\item 4 RRH dans le premier BBU, 2 RRH dans le deuxième BBU
	\item 5 RRH dans le premier BBU, 1 RRH dans le deuxième BBU
\end{itemize}
Les deux derniers cas sont symètriques donc inutile de les calculer, puisque nos BBU jouent
le même rôle (au moins dans cette simulation).\\
Nous avons créé des fonctions pour remplir les différentes matrices.
\begin{itemize}
	\item \textbf{Matrice G} : Matrice symètrique, diagonal nulle, et valeurs aléatoires entre 1 et 4
	\item \textbf{Matrice Y} : Matrice remplie avec la fonction \textit{pop\_Y(int BBU1)}
		le paramètre BBU1 représente le nombre de RRH dans le premier BBU, donc on peut
		calculer le nombre de RRH dans le deuxième.

\end{itemize}
\section{Formules}
\subsection{Coût}
\subsubsection{Original}
\begin{displaymath}
C(y_i)= \sum\limits_{b=1}^By_{ib}.\left(\alpha.\sum\limits_{j=1}^Ry_{jb}.n_j+(1-\alpha).(\sum\limits_{j\neq i}G_{ji}.(1-y_{jb})+N_0)\right)
\end{displaymath}
\subsubsection{Sur le code}
\begin{displaymath}
C(y_i)= \sum\limits_{b=1}^By_{ib}.(\alpha.A_b+(1-\alpha).B_b)
\end{displaymath}
\subsection{Débit}
\subsubsection{Original}
\begin{displaymath}
D(y,G)=W.\sum\limits_{i=1}^Ry_{ib}.\frac{log_2\left(1+\sum\limits_{j \neq i}^RG_{ji}.(1-y_{jb})\right)}{\sum\limits_{j=1}^Ry_{jb}.n_j}
\end{displaymath}
\subsubsection{Sur le code}
\begin{displaymath}
D(y,G)=W.\sum\limits_{i=1}^Ry_{ib}.\frac{log_2(1+F_b)}{G_b}
\end{displaymath}
\section{Algorithme}
\subsection{Coût}
Pour calculer le coût total moyen :
\begin{enumerate}
\item Il faut d'abord lancer la fonction \textit{init()} pour initialiser le nombre d'utilisateurs par RRH.
\item Ensuite la fonction \textit{cout(int nRRH)} calcule le coût en utilisant $A_b$ et 
$B_b$ pour le RRH numéro \textbf{nRRH}.
\item La fonction \textit{cout\_total()} calcule le coût total de toutes les RRH en utilisant la fonction \textit{cout(int nRRH)}.
\item La fonction \textit{cout\_moyen()} calcule le coût moyen de 20 cas différent, en changeant la matrice G à chaque fois à l'aide de \textit{pop\_G()}.
\item La fonction \textit{simulate(int BBU1)} ensuite calcule le coût moyen du cas avec \textbf{BBU1} de RRH dans le $1^{er}$ BBU et 6-\textbf{BBU1} RRH dans le $2^{eme}$ BBU
en changeant à chaque fois la matrice Y à l'aide de \textit{pop\_Y(int BBU1)}.
\item Finalement \textit{full\_sim()} effectue les simulations pour chaque cas et stock 
les résultats dans le vecteur \textbf{Resultat}.
\item Les résultats sont interprétés par la fonction \textit{disp\_res()}.
\end{enumerate} 
\end{document}
