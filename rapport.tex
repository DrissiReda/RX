\documentclass[a4paper,sans,titlepage,10pt]{article}

\usepackage[utf8]{inputenc}
\usepackage{amsmath}
\title{Projet RX}

\author { \textbf{Mohamed Reda DRISSI }
			\and
		   \textbf{Simon ROUVEL}}
\date{\textit{27 janvier 2018}}

\begin{document}

\maketitle

\tableofcontents
\clearpage
\section{Introduction}
Pour simuler tous les cas nous avons cherché les cas possibles :
\begin{itemize}
	\item 1 RRH dans le premier BBU, 5 RRH dans le deuxième BBU
	\item 2 RRH dans le premier BBU, 4 RRH dans le deuxième BBU
	\item 3 RRH dans le premier BBU, 3 RRH dans le deuxième BBU
	\item 4 RRH dans le premier BBU, 2 RRH dans le deuxième BBU
	\item 5 RRH dans le premier BBU, 1 RRH dans le deuxième BBU
\end{itemize}
Les deux derniers cas sont symètriques donc inutile de les calculer, puisque nos BBU jouent
le même rôle (au moins dans cette simulation).\\
Nous avons créé des fonctions pour remplir les différentes matrices.
\begin{itemize}
	\item \textbf{Matrice G} : Matrice symètrique, diagonal nulle, et valeurs aléatoires entre 1 et 4
	\item \textbf{Matrice Y} : Matrice remplie avec la fonction \textit{pop\_Y(int BBU1)}
		le paramètre BBU1 représente le nombre de RRH dans le premier BBU, donc on peut
		calculer le nombre de RRH dans le deuxième.
\end{itemize}
\section{Formules}
\subsection{Coût}
\subsubsection{Original}
\begin{displaymath}
C(y_i)= \sum_{b=1}^By_{ib}.\left(\alpha.\sum_{j=1}^Ry_{jb}.n_j+(1-\alpha).(\sum_{j\neq i}G_{ji}.(1-y_{jb})+N_0)\right)
\end{displaymath}
\subsubsection{Sur le code}
\begin{displaymath}
C(y_i)= \sum_{b=1}^By_{ib}.(\alpha.A_b+(1-\alpha).B_b)
\end{displaymath}
\end{document}
